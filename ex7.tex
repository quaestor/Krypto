\chead{Blatt 7}
\section*{Aufgabe 31}
Da das Ende des Textes sehr viele Platzhalter-Zeichen enthält lässt sich die
Schablone direkt aus dem letzten Block von $36$ Zeichen ablesen. Dechiffriert
wird mit folgendem Haskell-Programm:
\lstset{language=Haskell}
\begin{lstlisting}
import Data.List

text = "osDtre...is/s         e"

stencil = [(1,3),(2,1),(2,3),(2,6),(3,4),(4,5),(4,6),(5,2),(6,6)]

fullturn xs = concat
		[ sort $ xs
		, sort $ map (\(i,j) -> (j,7-i)) xs
		, sort $ map (\(i,j) -> (7-i,7-j)) xs
		, sort $ map (\(i,j) -> (7-j,i)) xs
		]

decipher :: String -> [(Int, Int)] -> String
decipher xs st
	| length xs >= 36 =
		[(take 36 xs) !! (flat s) | s <- (fullturn st)]
			++ decipher (drop 36 xs) st
	| otherwise = []
	where flat (i,j) = (i-1) * 6 + (j-1)

main = putStrLn $ decipher text stencil
\end{lstlisting}
Entschlüsselter Text:
\begin{verbatim}
Die Konstruktion von Drehrastern ist einfach: Man beschreibt einen Quadranten
des Feldes (mit gerader Zeilen- und Spaltenanzahl) mit Markierungen,
ueberlagert alle durch Rotation sich ergebenden Markierungen und waehlt dann
von jeder Markierung genau eine aus. Wuerfel und Drehraster wurden gern auf
militaerisch-taktischem Niveau verwendet. Im 1. Weltkrieg wurden vom deutschen
Heer ab Anfang 1917 Drehraster verwendet, sie hatten Bezeichnungen wie ANNA (5
x 5), BERTA (6 x 6), CLARA (7 x 7), DORA (8 x 8), EMIL (9 x 9) und FRANZ (10 x
10). Nach vier Monaten gab man das wieder auf. (F. L. Bauer, Entzifferte
Geheimnisse)
\end{verbatim}

\section*{Aufgabe 32}
\begin{enumerate}
	\item
	\lstset{language=Mathematica}
	\begin{lstlisting}
FromContinuedFraction/@
	(ToCharacterCode[{"ADVENT","SCHNEE","WEIHNACHTEN"}]-65)

  {21479/65448,62954/3409,574578367/25832394}
	\end{lstlisting}
	\item Mit dem Hinweis, dass der Schlüssel generierende Kettenbruch sehr
	viel Einsen enthält, also viele Bs, erkennt man durch Subtraktion des
	Bs von jedem Buchstaben schon relativ viel vom ursprünglichen Text,
	z.B. endet der Text auf \verb/HERRLICHKEIT/, durch weitere Suche
	erkennbarer Wörter konnte das Schlüsselwort \verb/CBBBDBBBBE/, das dem
	Kettenbruch $[2,1,1,1,3,1,1,1,1,4]$ entspricht, extrahiert werden. Der
	entschlüsselte Text lautet damit:

\begin{verbatim}
ESTREIBTDERWINDIMWINTERWALDEDIEFLOCKENHERDEWIEEINHIRTUNDMANCHETANNEAHNTWIEBALDES
IEFROMMUNDLICHTERHEILIGWIRDUNDLAUSCHTHINAUSDENWEISSENWEGENSTRECKTSIEDIEZWEIGEHIN
BEREITUNDWEHRTDEMWINDUNDWAECHSTENTGEGENDEREINENNACHTDERHERRLICHKEIT
\end{verbatim}
\end{enumerate}


\section*{Aufgabe 33}
\begin{enumerate}[(1)]
	\item Die Kettenbruchentwicklung von $\alpha$ ist:
	\[ \alpha = \frac{20101201}{10210102} = [ 1, 1, 31, 159, 40, 1, 15, 3 ] \]
	Die Näherungsbrüche sind somit:
	\begin{eqnarray*}
	1 &=& [1] \\
	2 &=& [1, 1] \\
	\frac{63}{32} &=& [1,1,31] \\
	\frac{10019 }{5089} &=& [1,1,31,159] \\
	\frac{400823}{203592} &=& [1,1,31,159,40] \\
	\frac{410842}{208681} &=& [1,1,31,159,40,1] \\
	\frac{6563453}{3333807} &=& [1,1,31,159,40,1,15] \\
	\end{eqnarray*}
		
	\item Für eine Zahl $\frac{p}{q} \in \Z$ mit $\left| \alpha -
	\frac{p}{q} \right| < \frac{1}{2 q^2}$ und $\text{ggT}(p,q) = 1$ gilt, dass
	$\frac{p}{q}$ in der Kettenbruchentwicklung von $\alpha$ als
	Näherungsbruch vorkommt.
	Somit sind durch obige Auflistung aller Näherungsbrüche alle möglichen Paare $(m,n) \in \mathbb{N} \times \mathbb{N}$ bestimmt, Überprüfung ergibt folgende Tupel, die die gewünschte Eigenschaft erfüllen:
	\[ (m,n) \in \{ (2,1), (63,32), (10019, 5089), (410842 ,208681 ), \]
	\[ (6563453 ,3333807 ), (20101201 ,10210102 )\} \]

	\item Die Ungleichungen 
	\[ \frac{1}{2 n^2} \leq \left| \alpha - \frac{m}{n} \right| < \frac{1}{n^2} \]
	erfüllen die übrigen Näherungsbrüche, wie man leicht nachprüfen kann:
	\[ (m,n) \in \{(1,1), (400823, 203592)\} \]
\end{enumerate}


\section*{Aufgabe 34}
Die Kettenbruchentwicklungen sind:
\[ \sqrt{11} = [3, [3, 6]] \]
\[ \sqrt{13} = [3, [1, 1, 1, 1, 6]] \]
\[ \sqrt{17} = [4, [8]] \]
\[ \sqrt{19} = [4, [2, 1, 3, 1, 2, 8]] \]
Hierbei bedeutet die Darstellung eines Kettenbruchs der Art $[a, [b]]$, dass
der Kettenbruch mit $b$ periodisch fortgesetzt wird.

Für $\sqrt{n^2 + 1}$ gilt:
\[ \alpha_0 = \sqrt{n^2 + 1}, \quad a_0 = n \]
\[ \alpha_1 = \frac{1}{\sqrt{n^2 + 1} - n} = \frac{\sqrt{n^2 + 1} + n}{{\sqrt{n^2 + 1}}^2 - n^2} = \sqrt{n^2 + 1} + n, \quad a_1 = 2n \]
\[ \alpha_2 = \frac{1}{\sqrt{n^2 + 1} + n - 2n} = \alpha_1, \quad a_2 = a_1 \]
Somit gilt mit obiger Darstellung:
\[ \sqrt{n^2 + 1} = [n, [2n]] \]
