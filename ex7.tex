\chead{Blatt 7}
\section*{Aufgabe 31}
Da das Ende des Textes sehr viele Platzhalter-Zeichen enthält lässt sich die
Schablone direkt aus dem letzten Block von $36$ Zeichen ablesen. Dechiffriert
wird mit folgendem Haskell-Programm:
\lstset{language=Haskell}
\begin{lstlisting}
import Data.List

text = "osDtre...is/s         e"

stencil = [(1,3),(2,1),(2,3),(2,6),(3,4),(4,5),(4,6),(5,2),(6,6)]

fullturn xs = concat
		[ sort $ xs
		, sort $ map (\(i,j) -> (j,7-i)) xs
		, sort $ map (\(i,j) -> (7-i,7-j)) xs
		, sort $ map (\(i,j) -> (7-j,i)) xs
		]

decipher :: String -> [(Int, Int)] -> String
decipher xs st
	| length xs >= 36 =
		[(take 36 xs) !! (flat s) | s <- (fullturn st)]
			++ decipher (drop 36 xs) st
	| otherwise = []
	where flat (i,j) = (i-1) * 6 + (j-1)

main = putStrLn $ decipher text stencil
\end{lstlisting}
Entschlüsselter Text:
\begin{verbatim}
Die Konstruktion von Drehrastern ist einfach: Man beschreibt einen Quadranten
des Feldes (mit gerader Zeilen- und Spaltenanzahl) mit Markierungen,
ueberlagert alle durch Rotation sich ergebenden Markierungen und waehlt dann
von jeder Markierung genau eine aus. Wuerfel und Drehraster wurden gern auf
militaerisch-taktischem Niveau verwendet. Im 1. Weltkrieg wurden vom deutschen
Heer ab Anfang 1917 Drehraster verwendet, sie hatten Bezeichnungen wie ANNA (5
x 5), BERTA (6 x 6), CLARA (7 x 7), DORA (8 x 8), EMIL (9 x 9) und FRANZ (10 x
10). Nach vier Monaten gab man das wieder auf. (F. L. Bauer, Entzifferte
Geheimnisse)
\end{verbatim}

\section*{Aufgabe 32}
\begin{enumerate}
	\item
	\lstset{language=Mathematica}
	\begin{lstlisting}
FromContinuedFraction/@
	(ToCharacterCode[{"ADVENT","SCHNEE","WEIHNACHTEN"}]-65)

  {21479/65448,62954/3409,574578367/25832394}
	\end{lstlisting}
	\item
\end{enumerate}
