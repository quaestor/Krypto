\chead{Blatt 2}
\section*{Aufgabe 6}
Durch Analyse der Buchstabenhäufigkeit kann man eine monoalphabetische
Substitutionschiffrierung erkennen. Als Schlüsselwort ergibt sich
\verb/KAESTNER/.

Der Lösungstext lautet:
\begin{verbatim}
EBEN BIN ICH AUS SALZBURG ZURUECKGEKOMMEN; NUN HOCK' ICH, MITTERNACHT IST
VORBEI, IN DER HOTELBAR UND TRINKE DAS VIELGELIEBTE `CHARLOTTENBURGER PILSNER',
WIE DIE FREUNDE DIE HERZHAFTE MISCHUNG AUS SEKT UND BIER GETAUFT HABEN. VOR
SECHS JAHREN WAR ICH ZUM LETZTENMAL IN SALZBURG. DOCH ALS KARL UND ICH HEUTE
MITTAG IM GARTEN DES STIEGLBRAEUS, HINTEN IN DER `WELT', SASSEN UND AUF DIE
STADT DER STREITBAREN UND KUNSTSINNIGEN ERZBISCHOEFE HINABSCHAUTEN, WAR ICH VON
NEUEM UEBERWAELTIGT. AUCH ANMUT KANN ERSCHUETTERN.
\end{verbatim}

\section*{Aufgabe 7}
Die Lösungsmatrix, die den Cyphertext in Klartext transformiert lautet:
\[ \renewcommand{\arraystretch}{1.4}
M^{-1} = 
\begin{bmatrix}
		-\frac{1}{6}& \frac{139}{48}& -\frac{11}{8}\\
		0& -\frac{19}{8}& \frac{5}{4}\\
		\frac{1}{6}& -\frac{43}{48}& \frac{3}{8}
\end{bmatrix} \]
Die Matrix erhält man durch Lösen des linearen Gleichungssystems
\[ M^{-1} \cdot A = B \]
wobei $A$ die $3\times 3$-Matrix aus den ersten $3$ Codevektoren ist und $B$
die korrespondierende $3\times 3$-Matrix, die man aus dem Hinweistext gewinnen
kann (\verb/TEXTBEGINN LIEBE PIA/, Caesar-Chiffre).
\begin{verbatim}
LIEBE PIA DAS SELTSAME AM SOMMER IST DASS ER SO SCHNELL VERGEHT LAS ICH GESTERN
IN EINEM BUCH VON ASTRID LINDGREN AUCH MEINE FERIEN SIND IRGENDWIE SCHNELL
VERGANGEN HAST DU LUST DASS WIR UNS IN DEN HERBSTFERIEN MAL SEHEN VIELE GRUESSE
DEINE FREUNDIN BIRGIT
\end{verbatim}

\section*{Aufgabe 8}
Für die Aufgabe müssen Gleichungssysteme gelöst werden (Lemma aus Skript
Kapitel 1, 2. Wechsel der Gitterbasis). Dies übernimmt für uns Mathematica
(Output ist eingerückt):
\lstset{language=Mathematica}
\begin{lstlisting}
(* Matrizen der Gitterbasen *)
L1 = {{27, 23, 8}, {49, 41, 10}, {5, 5, 2}};
L2 = {{4, -54, -16}, {13, -1, 8}, {3, 9, 5}};
L3 = {{2, -6, -1}, {12, 10, 11}, {7, -3, 3}};
L4 = {{1, 49, 10}, {11, 19, 10}, {6, 54, 12}};

(* Matrix der Unbekannten *)
X = Table[x[i, j], {i, 1, 3}, {j, 1, 3}];

(* Alle moeglichen Inklusionen, FindInstance loest das 
   LGS fuer ganzzahlige Matrixeintraege *)
{
(M12 = FindInstance[{X.L1 == L2}, Flatten[X], Integers]) != {},
(M13 = FindInstance[{X.L1 == L3}, Flatten[X], Integers]) != {},
(M14 = FindInstance[{X.L1 == L4}, Flatten[X], Integers]) != {},
(M21 = FindInstance[{X.L2 == L1}, Flatten[X], Integers]) != {},
(M23 = FindInstance[{X.L2 == L3}, Flatten[X], Integers]) != {},
(M24 = FindInstance[{X.L2 == L4}, Flatten[X], Integers]) != {},
(M31 = FindInstance[{X.L3 == L1}, Flatten[X], Integers]) != {},
(M32 = FindInstance[{X.L3 == L2}, Flatten[X], Integers]) != {},
(M34 = FindInstance[{X.L3 == L4}, Flatten[X], Integers]) != {},
(M41 = FindInstance[{X.L4 == L1}, Flatten[X], Integers]) != {},
(M42 = FindInstance[{X.L4 == L2}, Flatten[X], Integers]) != {},
(M43 = FindInstance[{X.L4 == L3}, Flatten[X], Integers]) != {}
}

	{False, False, True, False, True, False, False, 
	 True, False, False, False, False}

(X /. M14)[[1]]
(X /. M23)[[1]]

	{{-24, 6, 71}, {-2, 0, 13}, {-24, 6, 72}}
	{{2, -3, 11}, {1, -1, 7}, {4, -6, 23}}
\end{lstlisting}
Es existieren also nur die Relationen $\Lambda_4 \subset \Lambda_1$ sowie
$\Lambda_2 = \Lambda_3$ mit den Transformationsmatrizen
\[ M_{1,4} = \begin{bmatrix}-24&6&71\\-2&0&13\\-24&6&72\end{bmatrix},\quad\quad
M_{2,3} = \begin{bmatrix}2&-3&11\\1&-1&7\\4&-6&23\end{bmatrix} \]

\section*{Aufgabe 9}
\begin{lstlisting}
(* Gitterbasis *)
b1 = {90, -24, -52};
b2 = {41, 58, 82};
b3 = {21, 29, -97};

(* Quadrat der Norm || x b1 + y b2 + z b3 || *)
f[x_, y_, z_] := Expand[Plus @@ Map[#^2 &, x*b1 + y*b2 + z*b3]]

c =
Sqrt[Min[
   Minimize[{f[x, y, 1], -1 <= y <= 1, -1 <= x <= 1}, {x, y}][[1]], 
   Minimize[{f[x, 1, z], -1 <= x <= 1, -1 <= z <= 1}, {x, z}][[1]], 
   Minimize[{f[1, y, z], -1 <= y <= 1, -1 <= z <= 1}, {y, z}][[1]]
]]

	213907/Sqrt[8129129]
\end{lstlisting}

Wir erhalten also einen Wert von $\frac{213907}{\sqrt{8129129}} \approx 75$ für $c$.

\begin{lstlisting}

q = Floor[100/c];
Select[
  Flatten[
    (* Tabelle der Funktionswerte und zugehoeriger Argumente *)
    Table[{f[x, y, z], {x, y, z}}, {x, -q, q}, {y, -q, q}, {z, -q, q}],
    2],
  (* filtere die gewuenschten Ergebnisse *)
  #[[1]] <= 100^2 & 
]

	{{9595, {-1, 0, 1}}, {0, {0, 0, 0}}, {9595, {1, 0, -1}}}
\end{lstlisting}
Also gibt es in dem Gitter, das von $b_1, b_2, b_3$ generiert wird, genau drei
Vektoren $a$ mit $\|a\| \leq 100$, und zwar den Nullvektor, $b_3 - b_1$ und
$b_1 - b_3$.\\[0.8em]

\section*{Aufgabe 10}
Wir wählen die Vektoren $a_1 = (1, 1, 0, \dots, 0)$, $a_2 = (0, 1, 1, 0, \dots,
0)$, $\dots$, $a_{n-1} = (0, \dots, 0, 1, 1)$, $a_n = (0, \dots, 0, 0, 2)$.
Dies sind $n$ linear unabhängige Vektoren. Da für jeden Vektor $(x_1, \dots,
x_n)$ mit $x_1 + \cdots + x_n \equiv 0\mod 2$ die Darstellung $x_1 a_1 + (x_2 -
x_1) a_2 + \cdots + (x_{n-1} - x_{n-2}) a_{n-1} + (x_n - x_{n-1}) \frac{1}{2}
a_n$ existiert und die Koeffizientensummen dieser Vektoren je kongruent $0$
modulo $2$ sind und damit jede $\Z$-Linearkombination der $a_i$ ebenfalls einen
Vektor bildet, dessen Koeffizientensumme durch $2$ teilbar ist, ist $\Lambda =
\Z a_1 + \cdots + \Z a_n$ ein Gitter. Das Quadrat der Gitterdeterminante ist
\begin{eqnarray*}
\begin{vmatrix}
	2 & 1 & 0 & \cdots & \cdots & \cdots & 0 \\
	1 & 2 & 1 & \ddots &  &  & \vdots \\
	0 & \ddots & \ddots & \ddots & \ddots &  & \vdots \\
	\vdots & \ddots & \ddots & \ddots & \ddots & \ddots & \vdots \\
	\vdots & & \ddots & 1 & 2 & 1 & 0 \\
	\vdots & & & \ddots & 1 & 2 & 2 \\[0.8em] %wtf...
	0 & \cdots & \cdots & \cdots & 0 & 2 & 4
   \end{vmatrix} &=& 
 2 \cdot \begin{vmatrix}
	2 & 1 & 0 & \cdots & \cdots & \cdots & 0 \\
	1 & 2 & 1 & \ddots &  &  & \vdots \\
	0 & \ddots & \ddots & \ddots & \ddots &  & \vdots \\
	\vdots & \ddots & \ddots & \ddots & \ddots & \ddots & \vdots \\
	\vdots & & \ddots & 1 & 2 & 1 & 0 \\
	\vdots & & & \ddots & 1 & 2 & 2 \\[0.8em] %wtf...
	0 & \cdots & \cdots & \cdots & 0 & 1 & 2
   \end{vmatrix} = \\
 4 \cdot \begin{vmatrix}
	2 & 1 & 0 & \cdots & \cdots & \cdots & 0 \\
	1 & 2 & 1 & \ddots &  &  & \vdots \\
	0 & \ddots & \ddots & \ddots & \ddots &  & \vdots \\
	\vdots & \ddots & \ddots & \ddots & \ddots & \ddots & \vdots \\
	\vdots & & \ddots & 1 & 2 & 1 & 0 \\
	\vdots & & & \ddots & 1 & 2 & 1 \\[0.8em] %wtf...
	0 & \cdots & \cdots & \cdots & 0 & 1 & 1
   \end{vmatrix} &=& 4 \cdot
 \begin{vmatrix}
	2 & 1 & 0 & \cdots & \cdots & \cdots & 0 \\
	1 & 2 & 1 & \ddots &  &  & \vdots \\
	0 & \ddots & \ddots & \ddots & \ddots &  & \vdots \\
	\vdots & \ddots & \ddots & \ddots & \ddots & \ddots & \vdots \\
	\vdots & & \ddots & 1 & 1 & 0 & 0 \\
	\vdots & & & \ddots & 1 & 1 & 0 \\[0.8em] %wtf...
	0 & \cdots & \cdots & \cdots & 0 & 1 & 1
   \end{vmatrix} = \dots = 4
\end{eqnarray*}
Durch Zeilen- und Spaltenumformungen erhält man wie im letzten Schritt in der
unteren Ecke der Matrix zu sehen eine untere Dreiecksmatrix mit Determinante
gleich $1$. Somit ist die Gitterdeterminante gleich $2$.
