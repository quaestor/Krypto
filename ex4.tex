\chead{Blatt 4}
\section*{Aufgabe 16}
Hinweis: \verb/VIGENERECHIFFRE/

Da es sich um einen Brief handelt, lag nahe, dass der Text mit 
\verb/LIEBE BIRGIT/ beginnt; daraus ließ sich der Schlüssel \verb/HERBST/ für
die Vigenerechiffrierung ableiten.

Entschlüsselter Text:
\begin{verbatim}
LIEBE BIRGIT,
ES WAR SCHOEN, DASS WIR UNS IN DEN HERBSTFERIEN MAL WIEDER GETROFFEN
HABEN. ICH HOFFE, DU KANNST MEINE VIGENERE-CHIFFRIERTE NACHRICHT 
ENTSCHLUESSELN. UNSER LETZTES VERSCHLUESSELUNGSVERFAHREN MIT DER
MATRIZENMULTIPLIKATION WAR MIR ZU RECHENAUFWAENDIG. 
VIELE GRUESSE SENDET DIR DEINE FREUNDIN PIA
\end{verbatim}

\section*{Aufgabe 17}
\lstset{language=Mathematica}
\begin{lstlisting}
gedicht = 
  "CQY JNLG FUN CYZGFR ZUO FXYL EHJ MRDQBV.
  GYY IMKOFQ YKUUVSVZTEHO MJFR.
  FWV SBEKPEHW CVGHMRK NTTGKRZVMD,
  FG NACEI BTKVKQ BYOLIJNISP.

  TIZ FDMQ ZLUXSBQ QD FTXBFVCUEIT.
  KA MFU EKBXB XKPHMQ UDXA WCEGQO.
  DAC EKU CPA EID,SMJC FLADKQ YUKSW.
  COF, JGQ XBK KLT, RGE VYQXKQ.

  YFWBKG RSV ZUFDWZYH `LMWOZS TFV QCJ VKVTHU OWYEK' XZT SPEZY NFLGMPTH.";

(* erzeuge die Liste der x_i mit kleinsten
   Loesungen der diophantischen Gl. *)
xi = {};
For[i = 1000, i > 0, i++,
  If[Not[PrimeQ[i]], Continue[]];
  test = Reduce[{x^2 + y^2 == i && x > 0 && y > 0 && x <= y},
                {x, y}, Integers];
  If[test == False, Continue[]];
  xi = Append[xi, x /. ToRules[First[test]]];
  If[Length[xi] == 265, Break[]];
]

(* Entschluesselung *)
ret = "";
k = 1;
For[i = 1, i <= StringLength[gedicht], i++,
  a = StringTake[gedicht, {i}];
  If[a == " " || a == "." || a == "`" || a == "'" || a == ",", 
    (* Sonderzeichen sind nicht verschluesselt *)
    ret = ret <> a,
    (* else: *)
    {
      z = FromCharacterCode[
            Mod[((ToCharacterCode[a][[1]] - 65) - xi[[k]]), 26] + 65
          ];
      ret = ret <> z;
      k++;
    }
  ];
]

(* Ausgabe des entschluesselten Textes *)
ret

  "NUN GIBT DER HERBST DEM WIND DIE SPOREN.
  DIE BUNTEN LAUBGARDINEN WEHN.
  DIE STRASSEN AEHNELN KORRIDOREN,
  IN DENEN TUEREN OFFENSTEHN.

  DAS JAHR VERGEHT IN MONATSRATEN.
  ES IST SCHON WIEDER FAST VORBEI.
  UND WAS MAN TUT, SIND SELTEN TATEN.
  DAS,WAS MAN TUT, IST TUEREI.

  ANFANG DES GEDICHTS `HERBST AUF DER GANZEN LINIE' VON ERICH KAESTNER."
\end{lstlisting}

\section*{Aufgabe 18}
\begin{lstlisting}
(* (1) *)
p = 19^29 + 3030;
ii = i/.ToRules[First[
       Reduce[Mod[i^2,p] == Mod[-1,p] && i < p && i >= 0, i, Integers]
     ]]

  5422740134092326071960207930297296114

(* (2) *)
LatticeReduce[{{ii, 1}, {p, 0}}]

  {{3384068435690137203, -823348539292400500},
   {-823348539292400500, -3384068435690137203}}

(* (3) *)
a = Reduce[x^2 + y^2 == p, {x, y}, Integers];
Table[{x, y} /. ToRules[a[[i]]], {i, 1, Length[a]}] // TableForm

  -3384068435690137203   -823348539292400500
  -3384068435690137203    823348539292400500
   -823348539292400500  -3384068435690137203
   -823348539292400500   3384068435690137203
    823348539292400500  -3384068435690137203
    823348539292400500   3384068435690137203
   3384068435690137203   -823348539292400500
   3384068435690137203    823348539292400500
\end{lstlisting}

\newpage
\section*{Aufgabe 19}
Für $p=2$ erhalten wir als mögliche Basisvektoren $(1, 1)$ und $(2, 0)$. Die
Gitterdeterminante ist gleich $2$.

Sei $p \equiv 3\mod 4$. Dann ist $\Lambda_p$ kein Gitter, da für die
diophantische Gleichung keine Lösung existiert, denn Quadrate modulo $4$ sind
entweder äquivalent $0$ oder $1$, insbesondere niemals äquivalent $3$ modulo
$4$.

Im Fall $p \equiv 1\mod 4$ gilt nach Skript, dass $\Lambda_p =
\mathbb{Z}(i,1)+\mathbb{Z}(p,0)$ ein Gitter mit Determinante $p$ ist mit einem
$i^2 \equiv -1\mod p$.

\section*{Aufgabe 20}
\begin{enumerate}[(1)]
	\item \[ \left\vert \frac{u_i}{u_{i-1}} \right\vert = \left\vert
	\frac{u_{i-2}}{u_{i-1}} - \left\lfloor \frac{u_{i-2}}{u_{i-1}}
	\right\rceil \cdot \frac{u_{i-1}}{u_{i-1}} \right\vert \leq \frac{1}{2}\]
	
	\item Nach (1) gilt: \[ \vert u_i \vert \leq \frac{1}{2} \vert u_{i-1}
	\vert \leq \dots \leq \frac{1}{2^{i-1}} \vert u_1 \vert \]
	was sich induktiv zeigen lässt.
	
	\item Für $i = 1$ ist nichts zu zeigen. Nach Definition gilt
	\[ u_{i+1} = u_{i-1} - k\cdot u_i,\quad k = \left\lfloor
	\frac{u_{i-1}}{u_i} \right\rceil \in \Z \]
	und damit gilt mit folgendem Induktionsschritt die Behauptung für alle
	$i \geq 1$:
	\[ \Z u_{i+1} = \Z u_{i-1} - \Z k\cdot u_i = \Z u_{i-1} - \Z u_i + \Z u_i \]
	\[ \Leftrightarrow \Z u_i + \Z u_{i+1} = \Z u_{i-1} + \Z u_i \]
	
	\item folgt aus (3) mit $i = l$.
	
	\item Mit $u_0, u_1 \in \Z$ ist $\Z u_0 + \Z u_1$ ein Gitter in $\Z$.
	Da die Folge $\{|u_i|\}$ monoton fallend ist (2), insbesondere streng
	monoton fallend, solange die $u_i \neq 0$ sind, und alle $u_i$ in dem
	Gitter enthalten sind (3), muss es ein $l \in \mathbb{N}$ geben, ab dem
	alle $u_i = 0$ für $i \geq l$, da es sonst beliebig kleine Gittervektoren
	gäbe. Wegen (4) muss weiterhin gelten, dass $\text{ggT}(u_0, u_1) =
	|u_{l-1}|$ gilt, da $u_{l-1}$ zum einen ein Teiler von $u_0$ und $u_1$
	sein muss, weil sowohl $\Z u_1 \subset  \Z u_{l-1}$ als auch $\Z u_0
	\subset \Z u_{l-1}$ und zum anderen nur Vielfache des ggT durch
	ganzzahlige Linearkombinationen dargestellt werden können. 

	\item Hier gilt ebenfalls wegen (2), dass die Folge der $\{|u_i|\}$
	streng monoton fallend ist, solange $u_i \neq 0$ erfüllt ist; da jedoch
	$\Z u_0 + \Z u_1$ ein Gitter ist, und es einen kürzesten Gittervektor
	gibt und die $u_i$ in dem Gitter enthalten sind, dass ab einem $l \in
	\mathbb{N}$ alle $u_i = 0$ für $i \geq l$.

	\item Nach Aufgabe 2 ist $\Z u_0 + \Z u_1$ in diesem Fall kein Gitter.
	Insbesondere gibt es deshalb kein $u_{l-1}$, sodass $\Z u_0 + \Z u_1 =
	\Z u_{l-1}$ gilt; also kann kein $u_i = 0$ existieren, da sonst
	ebendiese Gleichung erfüllt wegen (4) wäre. Wegen (3) gilt $\Z u_0 + \Z u_1 = \Z
	u_{i-1} + \Z u_i $, somit ist $\Z u_i \in \Z u_0 + \Z u_1 = \Z + \Z
	\alpha$ und wegen (2) folgt $\lim_{i \rightarrow \infty} u_i = 0$, da
	alle $u_i \neq 0$.

	\item Bei der Gauß-Reduktion wird folgender Schritt durchgeführt:
	\[ b^* = b - \left\lfloor \frac{a \cdot b}{a \cdot a} \right\rceil \cdot a \]
	Mit $a = (u_{i-1}, 0)$ und $b = (u_{i-2}, 0)$ erhält man:
	\[ b^* = (u_i,0) = (u_{i-2},0) - \left\lfloor \frac{u_{i-1} \cdot
	u_{i-2}}{u_{i-1} \cdot u_{i-1} } \right\rceil \cdot (u_{i-1}, 0)
	(u_{i-2},0) - \left\lfloor \frac{u_{i-2}}{u_{i-1}} \right\rceil \cdot
	(u_{i-1}, 0)\]
	Im Allgemeinen gilt wegen (2), dass $u_i < u_{i-1}$ also auch $\|a\| <
	\|b\|$, somit müssen $a$ und $b$ nie vertauscht werden. Lediglich für
	den Fall $u_0 < u_1$ entspricht der Konstruktionsprozess nicht der
	Gauß-Reduktion. Ist dabei $\frac{u_0}{u_1} < \frac{1}{2}$, so ist $u_2
	= u_0$ und die Folge $(u_i, 0)$ entspricht ab $i = 1$ der
	Gauß-Reduktion. Ist dagegen $\frac{u_1}{u_0} < \frac{1}{2}$, dann ist
	$u_2 = u_0 - u_1$, während bei der Gauß-Reduktion $u_2 = u_1 - u_0$
	wäre. Dieses Vorzeichen ist der einzige Unterschied zum anderen Fall.
\end{enumerate}
