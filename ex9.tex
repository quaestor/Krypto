\chead{Blatt 9}
\section*{Aufgabe 41}
\section*{Aufgabe 42}
\begin{minipage}{0.45\textwidth}
\begin{tikzpicture}[scale=.7]
\coordinate (O) at (0, 0);
\coordinate (A) at (6, 0);
\coordinate (B) at (2, 7);
\coordinate (AB) at (8,7);

\node[left] at (O) {$(0,0)$};
\node[right] at (A) {$(6,0)$};
\node[left] at (B) {$(2,7)$};
\node[right] at (AB) {$(8,7)$};

\coordinate (C) at (3, 2.92857);
\coordinate (F) at (5, 4.07143);
\draw (O) -- (A) node(aline) {};
\draw (A) -- (AB) node(aabline) {};
\draw (AB) -- (B) node(babline) {};
\draw (B) -- (O) node(bline) {};

\draw[dashed] (3,0) -- node[fill=white] {$p$} (C);
\draw[dashed] (1,3.5) -- node[fill=white] {$l$} (C);
\draw[dashed] (C) -- node[fill=white] {$f$} (F);
\draw[dashed] (5,7) -- node[fill=white] {$a$} (F);
\draw[dashed] (7,3.5) -- node[fill=white] {$y$} (F);

\fill (C) circle (0.8mm) node[above] {$I$};
\fill (F) circle (0.8mm) node[below] {$R$};
\end{tikzpicture}
\vspace{5em}

(Hinweis: \verb/QREFPUYHRFFRYYVRTGVAQRETENSVX/)
\end{minipage}\hfill\begin{minipage}{0.40\textwidth}
%In der nebenstehenden Grafik kann man die Menge der naechstliegenden
%Gittervektoren fuer alle Vektoren v ablesen. Liegt der Vektor v innerhalb einer
%der vier Flaechen, so enthaelt die Menge C nur den angrenzenden Gitterpunkt.
%Liegt v auf einer der gestrichelten Linien, so enthaelt C die zwei
%Gitterpunkte, die zu den angrenzenden Flaechen gehoeren. Analog verhaelt es
%sich mit den Punkten I und R, hier gibt es drei naechstliegende Gitterpunkte.
EU ZC GI US HI NU TN GK NU IM SU OG YP CE YS PQ SU HF SU IR KU EG OE KU IM CV
QP KI KQ OF EN HG TI NU VK DN NM GL GM NQ IG SU PZ GI WY YV PG ZW GM NQ IG SU
WZ BH PG NK SU PR GH SZ IM CV UG MS VG ZW UC SU UN BG YA CW NE UN CV IM CV UR
GI ZY AY KI KG SU TQ UK ON QB GP SZ IR KU EG OE KU KY UP CV IM SU PQ OG NU UM
TI NU VK DN NM IL PU MS VY EN MO ZW WY ZP UK EU GI ZC GI VK KN OD RD KG FO NU
VY EU EN SU TQ UK ON QB GP SZ KY IR KU UX NE VK DN NM IL PU MS KU IR KU UV ZC
NU WY OE IG TU NU IM SU PA PH BK NU VK GK NG IG SU PQ YA VO ZW GI QB GP SZ KN
YX RD KW ED SZ IM SU IP SE NM SU CU PU CZ CV EB GI VK RC SZ KN ZC IG CU QP KI
KQ OF EN HG TI KU ER SZ NM IL PU MS KU 
\end{minipage}

\section*{Aufgabe 44}
\lstset{language=Mathematica}
\begin{lstlisting}
Mod[
 ExtendedGCD[9, 110][[2, 2]]*110*7 +
  ExtendedGCD[10, 99][[2, 2]]*99*7 +
  ExtendedGCD[11, 90][[2, 2]]*90*3,
 9*10*11]

  817

Mod[
 ExtendedGCD[17, 16*15][[2, 2]]*(16*15)*3 +
  ExtendedGCD[16, 17*15][[2, 2]]*(17*15)*10 +
  ExtendedGCD[15, 17*16][[2, 2]]*(17*16)*0,
 17*16*15]

  3930
\end{lstlisting}

\section*{Aufgabe 45}
\begin{enumerate}[(1)]
	\item Aus den Voraussetzungen für $c$ und die $b_i$ folgt sofort, dass
	\begin{eqnarray*}
		c &\equiv& a^3 \mod N_1\\
		c &\equiv& a^3 \mod N_2\\
		c &\equiv& a^3 \mod N_3
	\end{eqnarray*}
	Da $a < \min(N_1, N2, N_3)$ ist $a^3 < N = N_1 N_2 N_3$. Da der
	Chinesische Restesatz als Lösung für $c$ einen Wert kleiner $N$ liefert
	gilt wegen Äquivalenz, dass $c = a^3$.
	\item
\begin{lstlisting}
n1 = ...; b1 = ...;
n2 = ...; b2 = ...;
n3 = ...; b3 = ...;
a3 = Mod[
  ExtendedGCD[n1, n2*n3][[2, 2]]*(n2*n3)*b1 +
   ExtendedGCD[n2, n1*n3][[2, 2]]*(n1*n3)*b2 +
   ExtendedGCD[n3, n1*n2][[2, 2]]*(n1*n2)*b3,
  n1*n2*n3]

FromCharacterCode[(IntegerDigits[a3^(1/3), 100] /. (0 -> -32)) + 64]

  "BALD BEGINNEN DIE WEIHNACHTSFERIEN"
\end{lstlisting}
\end{enumerate}
