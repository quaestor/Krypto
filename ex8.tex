\chead{Blatt 8}
\section*{Aufgabe 36}
Durch eine kontrollierte Bruteforce-Attacke auf das Ende des Textes konnten wir
die Maske der Schablone identifizieren und mithilfe des Programms aus Aufgabe
31 entschlüsseln:
\begin{verbatim}
"DER DEZEMBER DAS JAHR WAR ALT HAT DUENNE HAAR IST GAR NICHT SEHR GESUND KENNT
SEINEN LETZTEN TAG DAS JAHR KENNT GAR DIE LETZTE STUND IST VIEL GESCHEHN WARD
VIEL VERSAEUMT RUHT BEIDES UNTERM SCHNEE WEISS LIEGT DIE WELT WIE HINGETRAEUMT
UND WEHMUT TUT HALT WEH NOCH WAECHST DER MOND NOCH SCHMILZT ER HIN NICHTS
BLEIBT UND NICHTS VERGEHT IST ALLES WAHN HAT ALLES SINN NUETZT NICHTS DASS MANS
VERSTEHT UND WIEDER STAPFT DER NIKOLAUS DURCH JEDEN KINDERTRAUM UND WIEDER
BLUEHT IN JEDEM HAUS DER GOLDENGRUENE BAUM WARST AUCH EIN KIND HAST SELBST
GEFUEHLT WIE HOLD CHRISTBAEUME BLUEHN HAST NUN DEN WEIHNACHTSMANN GESPIELT UND
GLAUBST NICHT MEHR AN IHN BALD TRIFFT DAS JAHR DER ZWOELFTE SCHLAG DANN DROEHNT
DAS ERZ UND SPRICHT DAS JAHR KENNT SEINEN LETZTEN TAG UND DU KENNST DEINEN
NICHT AUS DIE DREIZEHN MONATE VON ERICH KAESTNER       "
\end{verbatim}

\section*{Aufgabe 37}
\begin{lstlisting}
M = {{160734, 254024}, {936649, 1480280}};
iM = Inverse[M];

Decipher[b_, d_] := {x, y} /. ToRules[
	Select[Flatten[Table[
		Reduce[(b - {x, y}).iM == {i, j}
			&& -d < x < d && -d < y < d,
			{x, y},
			Integers],
		{i, 1, 26}, {j, 1, 26}]
	], Head[#] == And &][[1]]
];

B = ...; (* Cyphertext *)

S = Map[Decipher[#, 20] &, B];

FromCharacterCode[Flatten[Map[#.iM &, (B - S)]] + 64]

  ESISTEINSCHNEEGEFALLENUNDESISTDOCHNITZEITMANWIRFTMICHMITDENBALLENDER
  WEGISTMIRVERSCHNEITZ
\end{lstlisting}

\section*{Aufgabe 38}
\begin{lstlisting}
(* liefert die naechstliegenden Vektoren zum Vektor v im Gitter M 
   im Format {vektor, distanz} *)
cvp[ M_, v_] := Module[{a, b, r, s, g, d, gd},
  {a, b} = LatticeReduce[M];
  {{r, s}} = {r, s} /. Solve[v == r*a + s*b, {r, s}];
  r = Floor[r]; s = Floor[s];
  g = Flatten[Table[(r + i)*a + (s + j)*b, {i, 0, 1}, {j, 0, 1}], 1];
  d = Map[Norm[v - #]^2 &, g];
  gd = Thread[List[g, d]];
  Select[gd, #[[2]] == Min[d] &]
  ]

M = {{12419, -501}, {-64946, 2620}};

vs = {{934099/2, -37675/2}, {451434/19, -534521/19}, {3454/51, 
    5246/51}, {10081/83, 2540/83}};

Map[cvp[M, #] &, vs]

(* naechstliegende Vektoren zu...
	v1 *)
  ({{{467041, -18841}, 169/2}, {{467058, -18834}, 169/2}}
     (* v2 *)
   {{{23762, -28132}, 2105/361}}
     (* v3 *)
   {{{72, 98}, 109028/2601}}
     (* v4 *)
   {{{113, 27}, 582205/6889}, {{116, 38}, 582205/6889},
   	{{130, 34}, 582205/6889}})
\end{lstlisting}

\section*{Aufgabe 39}
\begin{lstlisting}
t1 = ToCharacterCode["Kryptographie und Gitter."];
t2 = ToCharacterCode["Gitter und Kryptographie."];

Conversion[elem_] := Block[{},
	If[elem == 32, Return[0]];
	If[elem == 46, Return[53]];
	If[64 < elem < 96, Return[elem-64]];
	If[96 < elem, Return[elem-70]]
	];

a = FromDigits[Map[Conversion, t1], 54]

  4472285973419119941786015904173246227895209

b = FromDigits[Map[Conversion, t2], 54]

  2896797551704571192014792263577413175460315

n=First[Select[Divisors[a-b],
	2^127 < # < 2^128 && 
	Quotient[Mod[a, #], 10^36] == 144 &]
	]

  300264612486096578953921029273076625202

f = Mod[a, n]

  144835051197494846316094180042972136621
\end{lstlisting}

\section*{Aufgabe 40}
\newcommand{\Zd}{\Z[\sqrt{-d}]}
\begin{enumerate}[(1)]
\item
	$\Zd$ ist Unterring, da $\Zd$ nicht die leere Menge ist und für alle
	$a, b \in \Zd$ gilt (Abgeschlossenheit):

	\[ a + b = x_a + y_a \sqrt{-d} + x_b + y_b \sqrt{-d} = (x_a + x_b) +
	(y_a + y_b) \sqrt{-d} \in \Zd \]
	\[ a \cdot b = (x_a + y_a \sqrt{-d}) \cdot (x_b + y_b \sqrt{-d}) = (x_a
	x_b - d y_a y_b) + (x_a y_b + x_b y_a) \sqrt{-d} \in \Zd \]

\item
	\[ \left| \frac{a}{b} - q \right| < 1 \Leftrightarrow \left|a -
	bq\right| = |r| < |b| \Leftrightarrow |r|^2 = w(r) < w(b) = |b|^2 \]

\item
	Betrachtet man $\Zd$ als Gitter und wählt die folgenden Elemente
	\begin{eqnarray*}
	x &+& y \sqrt{-d}\\
	(x+1) &+& y \sqrt{-d}\\
	(x+1) &+& (y+1) \sqrt{-d}\\
	x &+& (y+1) \sqrt{-d}
	\end{eqnarray*}
	die paarweise minimalen Abstand voneinander haben, so ist klar, dass
	der Punkt $(x+\frac{1}{2}) + (y+\frac{1}{2})\sqrt{-d} \in \mathbb{C}$
	maximalen Abstand vom Gitter hat:
	\[ d = w\left(\frac{1+\sqrt{-d}}{2}\right) = \frac{1}{4}+\frac{d}{4} \]
	Daraus folgt, dass der Abstand zwischen beliebigen Punkten $\frac{a}{b}
	\in \mathbb{C}$ mit $a, b \in \Zd$ und dem nächstliegenden Gittervektor
	$q$ höchstens $d$ beträgt:
	\[ \left|\frac{a}{b} - q\right| \leq \left|\frac{1+\sqrt{-d}}{2} -
	q\right| \]
	Damit, mit der Existenz eines nächsten Gittervektors $q$ und mit (2)
	folgt die Behauptung.
\item[(4), (5)]
	Für ein beliebiges $q = x + y \sqrt{-d}$ gilt
	\[ \left|\frac{1+\sqrt{-d}}{2} - q \right|^2 = \frac{1}{4}+\frac{d}{4}
	+ \underbrace{(x^2 - x) + d(y^2 - y)}_{\geq 0} \]
	Die Ungleichung $\sqrt{\frac{1}{4}+\frac{d}{4} + (x^2 - x) + d(y^2 -
	y)} < 1$ ist offensichtlich nur für $d \in \{1,2\}$ erfüllt. Nach (3)
	gilt also, dass $\Zd$ mit $d \geq 3$ nicht norm-euklidisch ist.
\end{enumerate}
