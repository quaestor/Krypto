\chead{Blatt 5}
\section*{Aufgabe 21}
Hinweis: \verb/TEXTANFANGLIEBEPIATEXTENDEBIRGIT/

Aus den im Hinweis angegebenen Transformationen ließ sich das Codewort
\verb/BACH/ für die Playfair-Chiffre ableiten: Die Buchstaben \verb/B/,
\verb/E/ und \verb/L/ müssen wegen der Transformation des zweiten
Buchstabenpaars \verb/EB/$\leftrightarrow$\verb/LE/  entweder im Codewort oder
in der gleichen Spalte der Transformationsmatrix sein. Mit ähnlichen
Überlegungen kann man die Position der Transformation
\verb/EP/$\leftrightarrow$\verb/KL/ in der Transformationsmatrix bestimmen und
daraufhin diese sukzessive mit den restlichen Buchstaben des Alphabets gefüllt
werden.

Entschlüsselter Text:
\begin{verbatim}
LI EB EP IA IS TD IR AU FG EF AL LE ND AS SA MS ON NT AG IN EI NE RW OC HE SC
HO ND ER ER ST EA DV EN TI ST IC HW UE RD EG ER NE MA LW IE DE RD AS WE IH NA
CH TS OR AT OR IU MV ON BA CH HO ER EN UE BR IG EN SK ON NT EI CH DE IN EL ET
ZT EN AC HR IC HT DI EV IG EN ER EV ER SC HL UE SX SE LT WA RE RF OL GR EI CH
EN TS CH LU ES SE LN IC HB IN GE SP AN NT OB DU ME IN EN PL AY FA IR CH IF FR
IE RT EN TE XT AU CH EN TS CH LU ES SE LN KA NX NS TV IE LE GR UE SX SE SE ND
ET DI RD EI NE FR EU ND IN BI RG IT 
\end{verbatim}

\section*{Aufgabe 22}
$d$ wurde mittels $q$ und $p$ aus Aufgabe 23 bestimmt. Für $k$ wurde $2$
gewählt.
\lstset{language=Mathematica}
\begin{lstlisting}
NN = 1991436079968060680956241683331340467648959931\
     9473433392606602383925001178815573;

b = 83424793841850446745114385885294325590533534318\
    60865256864804664771058882639227;

d = (1 + 2 (q - 1) (p - 1))/3

  1327624053312040453970827788887560311765250825267\ 
  5893031645722289910289566447147

a = PowerMod[b, d, NN]

  2215182005091205000701211919180504212609051820051\
  8000709202005180201190514

""<>(Map[FromCharacterCode[FromDigits[#, 10] + 64] &,
  Partition[IntegerDigits[a, 10], 2]] /. "@" -> " ")

  VORTEILE GAUSSREDUZIERTER GITTERBASEN
\end{lstlisting}

\section*{Aufgabe 23}
\begin{lstlisting}
NN = 1991436079968060680956241683331340467648959931\
     9473433392606602383925001178815573;
x1 = 3361927314620875058902199570867131105878;
y1 = 2934587795736682091568656460413386795367;
x2 = 3954174152664471768532159223441948595722;
y2 = 2068542378120596668138255705971813910233;

q = GCD[NN, x1 x2 + y1 y2]

  2345123709875208947523894129342145034693

p = GCD[NN, x1 y2 + y1 x2]

  8491816749718636190495054930224684110161
\end{lstlisting}

\section*{Aufgabe 24}
\begin{enumerate}[(1)]
	\item Es folgt sofort, dass $x^2 \equiv p\bmod 3$ ist; da Quadratzahlen
	modulo $3$ lediglich äquivalent $1$ oder $0$ sein können, jedoch $p
	\not\equiv 0\bmod 3$ für alle Primzahlen, gilt die Behauptung.
	
	\item $\Z_p^*$ ist zyklische Gruppe, also gibt es ein Generatorelement
	$g \in \Z_p^*$, das die Gruppe erzeugt:
	\[ \Z_p^* = \left\{g^k\ \vert\ k \in \{1,2,\dots,p-1\}\right\} \]
	Insbesondere gilt dann für alle $k \in \{1,\dots,p-2\}$, dass $g^k
	\not\equiv 1\bmod p$.

	Da $p \equiv 1\bmod 3$ gilt $3 \vert p-1$. Somit ist die Behauptung
	wohldefiniert.

	\item Da $z^3 \equiv g^{p-1} \equiv 1\bmod p$, da $p-1$ die Ordnung von
	$\Z_p^*$ ist und weiterhin $1-z \not\equiv 0\bmod p$ gilt:
	\[ (1-z)(z^2+z+1) \equiv 1-z^3 \equiv 0\mod 3\]
	Daraus folgt die Behauptung, da $\Z_p$ ein Körper ist (und damit keine
	Nullteiler existieren).

	\item Mit $w \equiv 2 z + 1 \bmod p$ und $k ( z^2 + z + 1) \equiv 0 \bmod p$ erhält man
	\[ w^2 \equiv 4 z^2 + 4 z + 1 \equiv 4 (z^2 + z + 1) - 3 \equiv -3 \mod p\]

	\item Mit 
	\begin{lstlisting}
p = 29^31 + 80;
w = w/.ToRules[Reduce[Mod[w^2, p] == p - 3 && 0 <= w <= p, w, Integers]]
  {259680314704997257972162448645375431635114923,
   1899743740103581306194335079943409130737482586}
	\end{lstlisting}
	erhält man Zahlen, die die gewünschte Eigenschaft $w^2 \equiv -3 \bmod
	p$ erfüllen.
	Durch Gauß-Reduktion wird ein Vektor $a = (u, v \sqrt{3})$ bestimmt:
\begin{lstlisting}
GaussReduktion[aa_, bb_] := 
  Module[{a = aa, b = bb}, 
    If[b.b < a.a, {a, b} = {b, a}];
    While[True,
      If[a.a == 0, Break[]];
      b = b - Round[a.b/a.a]*a;
      If[b.b >= a.a, Break[], {b, a} = {a, b}];
    ];
    {a, b}
  ]
  a1 = {w[[1]], Sqrt[3]};
  bx = {p, 0};
GaussReduktion[a1, bx]
  {{42007181439066575679179, -11472005961672078620034 Sqrt[3]}, 
   {-34416017885016235860102, -42007181439066575679179 Sqrt[3]}}
\end{lstlisting}
	Damit hat man Zahlen $u = 42007181439066575679179, v =
	-11472005961672078620034$, die die Gleichung $u^2 + d v^2 = p$
	erfüllen. Die Lösungsmenge der diophantischen Gleichung sind alle
	Kombinationen von $(x,y) = (\pm u, \pm v)$.
\end{enumerate}

\section*{Aufgabe 25}
\begin{enumerate}[(1)]
	\item 
	\item 
	\item
	\item
\end{enumerate}
