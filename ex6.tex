\chead{Blatt 6}
\section*{Aufgabe 26}
In der letzten Zeile des Textes ließ sich erkennen, dass dieselbe Grußformel
wie in den letzten Briefen verwendet wurde. Anhand der dadurch bekannten
Umformungen war es ein Leichtes, den Schlüssel \verb/ADVENTSZEIT/ zu finden.

Entschlüsselter Text:
\begin{verbatim}
LI EB EB IR GI TW EI LD UI ND EI NE RL ET ZT EN NA CH RI CH TD AS WE IH NA CH
TS OR AT OR IU MV ON BA CH AN GE SP RO CH EN HA ST HA BE IC HM IC HE TW AS UM
GE SC HA UT KU RZ VO RW EI HN AC HT EN GI BT ES VO ND ER ER LA NG ER DR EI FA
LT IK EI TS KA NT OR EI DI EX ER ST EH AE LF TE ZU HO ER EN WI RK OE NX NE NG
ER NG EM EI NS AM HI NG EH EN UE BR IG EN SD EN KE IC HD AS SU NS ER IE TZ IG
ES PL AY FA IR PA SX SW OR TA DV EN TS ZE IT EI NI GE RM AS SE NS IC HE RI ST
VI EL EG RU ES SE SE ND ET DI RD EI NE FR EU ND IN PI AX 
\end{verbatim}

\section*{Aufgabe 27}
Das folgende (zugegebenermaßen unübersichtliche und unverständliche)
Haskell-Programm führt den \textsc{Kasiski}-Test mit dem gegebenen Text durch:
\lstset{language=Haskell}
\begin{lstlisting}
import Data.List
import Data.Function

-- Der verschluesselte Text
chiff = "YEX...BAC"

-- Haeufigkeitsanalyse
countdooku xs = sortBy (compare `on` snd) $
		map (\x -> (head x, length x)) $ group $ sort xs

-- Sortiert Textbloecke der Laenge 'n' nach Auftrittshaeufigkeit
samefags n xs = countdooku $ samefags' n xs
samefags' n (x:xs) = take n (x:xs) : samefags' n xs
samefags' n _ = []

-- Liefert die Positionen (i und j) der haeufigsten Textbloecke
substr xs ys = substr' xs ys 0
substr' xs (y:ys) n | take m (y:ys) == xs = n : substr' xs ys (n+1)
		    | otherwise = substr' xs ys (n+1)
			where m = length xs
substr' _ _ _ = []

-- Berechnet den ggT der Differenzen (j - i)
keylength = foldl1 gcd $
	map (\x -> (x!!1) - (x!!0)) $
	map (flip substr chiff)  $
	map fst $
	filter (\x -> snd x == 2) $
	samefags 5 chiff

-- Sortiert die Buchstaben des verschluesselten Textes nach Position
-- modulo 7 (die Schluessellaenge, die mit vorhergehenden Funktionen 
-- ermittelt wurde)
seventhson xs = sonofaseventhson xs [[],[],[],[],[],[],[]]
sonofaseventhson (son1:son2:son3:son4:son5:son6:son7:xs)
		 (bro1:bro2:bro3:bro4:bro5:bro6:bro7:[]) = 
			sonofaseventhson xs
				[son1:bro1,son2:bro2,son3:bro3,
				 son4:bro4,son5:bro5,son6:bro6,
				 son7:bro7]
sonofaseventhson [] ys = ys
sonofaseventhson _ ys = ys

-- gibt das vermutete Codewort aus
main = map (\x -> chr (x+65)) $
	map (\x -> (ord (fst x) - 65) - 4) $
	map last $ map countdooku (seventhson chiff)
\end{lstlisting}
Die main-Funktion liefert das Codewort \verb/UMBERTO/ (der Vorname des
berühmten Autors \verb/ECO/). Entschlüsselung des Textes folgt analog zu
Aufgabe 16:
\begin{verbatim}
ESWAREINKLARERSPAETHERBSTLICHERMORGENGEGENENDENOVEMBERINDERNACHTHATTEESEINWENIGG
ESCHNEITUNDSOBEDECKTEEINFRISCHERWEISSERSCHLEIERKAUMMEHRALSZWEIFINGERHOCHDENBODEN
NOCHBEIDUNKELHEITGLEICHNACHLAUDESHATTENWIRTALABWAERTSINEINEMDORFDIEMESSEGEHOERTD
ANNWARENWIRAUFGEBROCHENUMBEIMERSTENTAGESLICHTINDIEBERGEZUGEHENALSWIRDENSTEILENPF
ADERKLOMMENDERSICHDIEHAENGEHINAUFWANDSAHICHZUMERSTENMALDIEABTEINICHTIHREMAUERNUE
BERRASCHTENMICHSIEGLEICHENDENANDERENDIEICHALLERORTENINDERCHRISTLICHENWELTGESEHEN
SONDERNDIEMASSIGKEITDESSENWASSICHSPAETERALSDASAEDIFICIUMHERAUSSTELLENSOLLTEESWAR
EINACHTECKIGERBAUDERAUSDERFERNEZUNAECHSTWIEEINVIERECKAUSSAHDIEHOECHSTVOLLENDETEF
ORMAUSDRUCKDERBESTAENDIGKEITUNDUNEINNEHMBARKEITDERSTADTGOTTESSEINESUEDFLANKERAGT
EHOCHUEBERDASPLATEAUDERABTEIWAEHRENDDIENORDMAUERNUNMITTELBARAUSDEMBERGHANGZUWACH
SENSCHIENENGLEICHSCHRAEGIMFELSVERWURZELTENBAEUMENVONUNTENGESEHENSCHIENESGERADEZU
ALSVERLAENGERTESICHDERFELSENZUMHIMMELUMINEINERGEWISSENHOEHEOHNESICHTBARENWANDELI
NFAERBUNGUNDSTOFFZUMMAECHTIGENTURMZUWERDENEINWERKVONRIESENHANDGESCHAFFENINGROESS
TERVERTRAUTHEITMITHIMMELUNDERDEDREIFENSTERREIHENSKANDIERTENDENTRIPELRHYTHMUSDESA
UFBAUSDERGESTALTDASSWASPHYSISCHALSQUADRATAUFDERERDESTANDSICHSPIRITUELLALSDREIECK
ZUMHIMMELERHOB
\end{verbatim}

\section*{Aufgabe 28}
\lstset{language=Mathematica}
\begin{lstlisting}
(* (1) *)
{p,q}=Map[First,FactorInteger[n]]

  {50821,196769}

phin = (p-1)(q-1)

  9999749760

(* (2) *)
e=First@First@Select[Table[{ee,GCD[ee,phin]},{ee,2,100}],#[[2]]==1&]

  17

(* (3) *)
d=ExtendedGCD[e,phin][[2,1]]

  4705764593

(* (4) *)
PowerMod[IntegerDigits[FromDigits[(ToCharacterCode["ERSTER ADVENT"]/. 32->64)-64,100],10^5],e,n]

  {6098921088,9250330136,1844742,6215066524,14485995}

(* (5) *)
FromCharacterCode[
  (IntegerDigits[
    FromDigits[PowerMod[{6057562331,3316356015,5652398482,6086215719},d,n],10^10]
   ,100]+64)/. 64->32]

  BALD IST WEIHNACHTEN
\end{lstlisting}

\section*{Aufgabe 29}
\begin{enumerate}[(1)]
	\item
	Folgendes Haskell-Programm (diesmal übersichtlich und verständlich) berechnet
	die Kettenbruchzerlegung einer rationalen Zahl $\frac{p}{q}$ nach der Methode
	aus dem Skript:
	\begin{lstlisting}
cont'dfrac :: Integral a => a -> a -> [a]
cont'dfrac p 0 = []
cont'dfrac p q = r : cont'dfrac q (p - r * q)
        where   r = p `div` q 

a = 6994
b = 1315
c = 2804
d = 477
numbers = [(a,b),(-a,b),(b,a),(-b,a),(c,d),(-c,d),(d,c),(-d,c)]

main = map (uncurry cont'dfrac) numbers

  [[5,3,7,4,2,6],[-6,1,2,7,4,2,6],[0,5,3,7,4,2,6],[-1,1,4,3,7,4,2,6],
   [5,1,7,4,2,6],[-6,8,4,2,6],[0,5,1,7,4,2,6],[-1,1,4,1,7,4,2,6]]
	\end{lstlisting}
	\item 
	\item Ist in der Kettenbruchzerlegung $\alpha = [a_0, a_1, \dots]$ die
		Zahl $a_0 = 0$, so ist $\frac{1}{\alpha} = [a_1, a_2, \dots]$.
		Ist $a_0 \not= 0$, so fügt man vorne eine $0$ ein. Begründung:
		\[ \frac{1}{\alpha} = 
			\begin{cases}
				\left(\frac{1}{a_1+\frac{1}{a_2+\dots}}\right)^{-1} =
				a_1 + \frac{1}{a_2+\frac{1}{a_3+\dots}} &a_0 = 0 \\
				0 + \frac{1}{a_0+\frac{1}{a_1+\dots}} &a_0 \not= 0
			\end{cases}
				\]
\end{enumerate}

\section*{Aufgabe 30}
\begin{enumerate}[(1)]
\item Lösung mit Mathematica:
\begin{lstlisting}

n = 4205699657;
p=56311;
q=74687;
e={7,841113733,3911015213,3973765663};
phin=(p-1)*(q-1);

(* (1) *)
d=Map[Mod[#[[2,1]],phin]&,ExtendedGCD[e,phin]]
  {600795523,2523341197,257,127}

k=(e*d-1)/phin
 {1,504668240,239,120}
\end{lstlisting}
\item
Diese Teilaufgabe lösen wir mithilfe des Haskell-Programms aus Aufgabe 29,
wobei die Variablen $phin, e, d$ und $n$ mit den Werten oder Listen aus (1)
initialisiert wurden.
\lstset{language=Haskell}
\begin{lstlisting}
map (uncurry cont'dfrac) $ zip k d

  [[0,600795523],[0,4,1,168222745,1,2],
   [0,1,13,3,1,1,2],[0,1,17,7]]

map (uncurry cont'dfrac) $ zip e (repeat phin)

  [[0,600795522,1,6],[0,4,1,168222745,1,1,2],
   [0,1,13,3,1,1,2,16364080],[0,1,17,6,1,33114713]]

map (uncurry cont'dfrac) $ zip e (repeat m)
	where m = n - floor(sqrt(4 * fromInteger n))

  [[0,600795707,1,6],[0,5,652026,6,1,2,6,10],
   [0,1,13,3,1,1,1,1,52,3,1,4,1,8,2,18,39],
   [0,1,17,7,212,1,20,3,10,235]]

map (uncurry cont'dfrac) $ zip e (repeat n)

  [[0,600814236,1,2,2],[0,5,6421,9,3,2,4,1,7,1,4,2,1,2],
   [0,1,13,3,1,2,9,2,1,26,48,1,3,1,2,6,1,6],
   [0,1,17,7,1,1,27,1,1,90,3,2,1,9,10,3]]

\end{lstlisting}
\end{enumerate}
